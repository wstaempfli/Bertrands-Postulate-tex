
% -------------------------------------------------------------------
% Preamble
% -------------------------------------------------------------------
\documentclass[12pt]{beamer}
\setbeamertemplate{footline}[frame number]
\usepackage[most]{tcolorbox} % Include tcolorbox package
\usepackage{xcolor} 
\usepackage{amsmath} % For align* environment

% Choose a different theme and color scheme compared to the original
\usetheme{default} 
\usecolortheme{whale}

\usepackage[utf8]{inputenc}
\usepackage[T1]{fontenc}
\usepackage{lmodern}  % for better fonts
\usepackage{tikz}
\usepackage{graphicx}

%-------------------------------------------------------------------
% Color Definitions (Add to Preamble)
%-------------------------------------------------------------------
\definecolor{PrimaryBlue}{RGB}{0, 76, 151}    % Dark blue from whale theme
\definecolor{AccentOrange}{RGB}{237, 125, 49} % Warm contrast color
\definecolor{LightGray}{RGB}{245, 245, 245}   % For subtle backgrounds
\definecolor{DarkGray}{RGB}{64, 64, 64}       % For text and frames

%-------------------------------------------------------------------
% Revised Bertrand's Box
%-------------------------------------------------------------------
\newtcolorbox{bertrandbox}{
  colback=LightGray,
  colframe=DarkGray,
  boxrule=0.6pt,
  arc=2pt,
  fontupper=\color{DarkGray},
  left=6pt,
  right=6pt,
  top=4pt,
  bottom=4pt
}

% Add to preamble
\usepackage{calc} % For \dimexpr calculations
\newcommand{\resultarrow}{\hfill\ensuremath{\rightarrow\ }} % Consistent arrow format

\newcommand{\bertrandpostulat}{
  \begin{bertrandbox}
    \small For every $n \geq 1$ there is some prime number $p$ with $n < p \leq 2n$
  \end{bertrandbox}
  %\vspace{1em}
}
%-------------------------------------------------------------------
%Header Format
%-------------------------------------------------------------------

\newcommand{\header}[1]{%
% Use the frametitle foreground color and desired size
{% Group to keep color change local
  \usebeamercolor[fg]{frametitle}% Get the correct text color
  \scriptsize % Set desired font size
  #1% Use provided argument
}%
}

%-------------------------------------------------------------------
%PascalTriangle - Normal
%-------------------------------------------------------------------
\newcommand{\PascalTriangle}[2][odd node/.style={text=gray!60, inner sep=1.5pt}]{%
% #1 = TikZ style definition(s) (optional, defaults to setting 'odd node' style with gray text)
% #2 = Max Row Number (mandatory)
\begin{tikzpicture}[
    scale=0.75, % Default scale, adjust as needed
    transform shape,
    base node/.style={inner sep=1.5pt, text=black}, % Base style for ALL nodes (explicitly black text)
    #1 % Apply the user/default style definitions here (defines/overrides 'odd node')
  ]
  % Define spacing
  \def\hdist{1.3} % Horizontal distance factor
  \def\vdist{0.9} % Vertical distance between rows

  % Loop through rows (n from 0 up to the provided argument #2)
  \foreach \n in {0,...,#2} {
    % Loop through elements in the row (k from 0 to n)
    \foreach \k in {0,...,\n} {
      % Calculate node position
      \pgfmathsetmacro{\xpos}{(\k - \n/2) * \hdist}
      \pgfmathsetmacro{\ypos}{-\n * \vdist}

      % Place the node, applying 'odd node' style conditionally
      \ifodd\n % TeX primitive to check if \n is odd
        \node [base node] at (\xpos, \ypos) {$\binom{\n}{\k}$}; % Use gray style
      \else
        \node [base node] at (\xpos, \ypos) {$\binom{\n}{\k}$}; % Use base (black) style
      \fi
    }
  }

  % Pre-calculate the y-coordinate for the ellipsis
  \pgfmathsetmacro{\ellipsisypos}{-(#2 + 1) * \vdist}
  % Add ellipsis below the last row using the base (black) style
  \node [base node] at (0, \ellipsisypos) {$\ldots$};

\end{tikzpicture}% End TikZ picture
}

%-------------------------------------------------------------------
%PascalTriangle - highligh rows
%-------------------------------------------------------------------
\newcommand{\PascalTriangleHighlight}[2][odd node/.style={text=gray!60, inner sep=1.5pt}]{%
% #1 = TikZ style definition(s) (optional, defaults to setting 'odd node' style with gray text)
% #2 = Max Row Number (mandatory)
\begin{tikzpicture}[
    scale=0.75, % Default scale, adjust as needed
    transform shape,
    base node/.style={inner sep=1.5pt, text=black}, % Base style for ALL nodes (explicitly black text)
    #1 % Apply the user/default style definitions here (defines/overrides 'odd node')
  ]
  % Define spacing
  \def\hdist{1.3} % Horizontal distance factor
  \def\vdist{0.9} % Vertical distance between rows

  % Loop through rows (n from 0 up to the provided argument #2)
  \foreach \n in {0,...,#2} {
    % Loop through elements in the row (k from 0 to n)
    \foreach \k in {0,...,\n} {
      % Calculate node position
      \pgfmathsetmacro{\xpos}{(\k - \n/2) * \hdist}
      \pgfmathsetmacro{\ypos}{-\n * \vdist}

      % Place the node, applying 'odd node' style conditionally
      \ifodd\n % TeX primitive to check if \n is odd
        \node [base node, odd node] at (\xpos, \ypos) {$\binom{\n}{\k}$}; % Use gray style
      \else
        \node [base node] at (\xpos, \ypos) {$\binom{\n}{\k}$}; % Use base (black) style
      \fi
    }
  }

  % Pre-calculate the y-coordinate for the ellipsis
  \pgfmathsetmacro{\ellipsisypos}{-(#2 + 1) * \vdist}
  % Add ellipsis below the last row using the base (black) style
  \node [base node] at (0, \ellipsisypos) {$\ldots$};

\end{tikzpicture}% End TikZ picture
}

%-------------------------------------------------------------------
%PascalTriangle - highligh middle elements
%-------------------------------------------------------------------
\newcommand{\PascalTriangleHighlightMid}[2][highlight node/.style={text=PrimaryBlue}]{%
 \begin{tikzpicture}[
     scale=0.75,
     transform shape,
     base node/.style={inner sep=1.5pt, text=black},
     odd node/.style={text=gray!60, inner sep=1.5pt}, % Original gray style
     #1 % Custom styles (highlight node or override odd node)
   ]
   % Maintain original spacing
   \def\hdist{1.3}
   \def\vdist{0.9}

   \foreach \n in {0,...,#2} {
     \foreach \k in {0,...,\n} {
       \pgfmathsetmacro{\xpos}{(\k - \n/2) * \hdist}
       \pgfmathsetmacro{\ypos}{-\n * \vdist}

       % First apply odd row styling
       \ifodd\n
         \node [base node, odd node] at (\xpos, \ypos) {$\binom{\n}{\k}$};
       \else
         % Then check for middle node in even rows
         \pgfmathtruncatemacro{\twok}{2*\k}
         \ifnum\twok=\n
           \node [base node, highlight node] at (\xpos, \ypos) {$\binom{\n}{\k}$};
         \else
           \node [base node] at (\xpos, \ypos) {$\binom{\n}{\k}$};
         \fi
       \fi
     }
   }

   % Maintain original ellipsis placement
   \pgfmathsetmacro{\ellipsisypos}{-(#2 + 1) * \vdist}
   \node [base node] at (0, \ellipsisypos) {$\ldots$};
 \end{tikzpicture}%
}
%-------------------------------------------------------------------
%PascalTriangle - Sum
%-------------------------------------------------------------------
\newcommand{\PascalHighlightSum}[2][highlight node/.style={text=PrimaryBlue}]{%
 \begin{tikzpicture}[
     scale=0.75,
     transform shape,
     base node/.style={inner sep=1.5pt, text=black},
     odd node/.style={text=gray!60, inner sep=1.5pt},
     sum node/.style={base node, anchor=west}, % Style for sum equation
     #1
   ]
   \def\hdist{1.3}
   \def\vdist{0.9}

   \foreach \n in {0,...,#2} {
     % Draw binomial coefficients
     \foreach \k in {0,...,\n} {
       \pgfmathsetmacro{\xpos}{(\k - \n/2) * \hdist}
       \pgfmathsetmacro{\ypos}{-\n * \vdist}

       \ifodd\n
         \node [base node, odd node] at (\xpos, \ypos) {$\binom{\n}{\k}$};
       \else
         \pgfmathtruncatemacro{\twok}{2*\k}
         \ifnum\twok=\n
           \node [base node, highlight node] at (\xpos, \ypos) {$\binom{\n}{\k}$};
         \else
           \node [base node] at (\xpos, \ypos) {$\binom{\n}{\k}$};
         \fi
       \fi
     }
     
     % Add plus signs between nodes (only if row has >1 element)
     \ifnum\n>0
       \foreach \k in {0,...,\numexpr\n-1} {
         \pgfmathsetmacro{\plusx}{(\k + 0.5 - \n/2) * \hdist}
         \pgfmathsetmacro{\ypos}{-\n * \vdist}
         \ifodd\n
           \node [base node, odd node] at (\plusx, \ypos) {$+$};
         \else
           \node [base node] at (\plusx, \ypos) {$+$};
         \fi
       }
     \fi
     
     % Add row sum equation
     \pgfmathsetmacro{\sumx}{(\n/2 + 0.7) * \hdist}
     \node [sum node] at (\sumx, -\n*\vdist) {$ = \enspace 2^{\n}$};
   }

   % Maintain ellipsis placement
   \pgfmathsetmacro{\ellipsisypos}{-(#2 + 1) * \vdist}
   \node [base node] at (0, \ellipsisypos) {$\ldots$};
 \end{tikzpicture}%
}
%-------------------------------------------------------------------
% Central Question Box (Subtle + Centered + Dynamic Sizing)
%-------------------------------------------------------------------
\newtcolorbox{centralquestionbox}{
  colback=LightGray!20,
  colframe=PrimaryBlue!30,
  boxrule=0.8pt,
  arc=2pt,
  left=8pt,
  right=8pt,
  top=8pt,
  bottom=8pt,
  fontupper={\sffamily\normalsize\color{DarkGray!80}\centering}, % Centered text
  shadow={0.5pt}{-0.5pt}{0pt}{0pt}{LightGray!80},
  before upper={\vspace*{0.2em}},
  after upper={\vspace*{0.2em}},
  overlay={
    \node[anchor=north east] at (frame.north east) 
    {\textcolor{LightGray!60}{\fontsize{18}{18}\selectfont ?}};
  }
}

\newcommand{\centralquestion}[1]{
  \begin{centralquestionbox}
    #1
  \end{centralquestionbox}
}
%-------------------------------------------------------------------
% Result Box Styling
%-------------------------------------------------------------------
\newtcolorbox{resultbox}{
  colback=AccentOrange!5,
  colframe=AccentOrange!80,
  boxrule=0.8pt,
  arc=2pt,
  fontupper=\color{DarkGray}
}

\newcommand{\finalresult}[1]{
  \vspace{-0.5em}
  \begin{resultbox}
    \small #1
  \end{resultbox}
  \vspace{0.5em}
}

%-------------------------------------------------------------------
% Core Box Styling
%-------------------------------------------------------------------
\newtcolorbox{corebox}{
  colback=PrimaryBlue!5,
  colframe=PrimaryBlue!80,
  boxrule=0.8pt,
  arc=2pt,
  fontupper=\color{DarkGray}
}

\newcommand{\corecontent}[1]{
  \vspace{-0.5em}
  \begin{corebox}
    \small #1
  \end{corebox}
  \vspace{0.5em}
}

%-----------------------------------------------------------------
%remove toolbox at bottom right
%-----------------------------------------------------------------
\setbeamertemplate{navigation symbols}{}
\setbeamertemplate{footline}{}


%-------------------------------------------------------------------
%Document starts
%-------------------------------------------------------------------

\title[Bertrand's Postulate]{Bertrand's Postulate}
\subtitle{For every $n \geq 1$ there is some prime $p$ with $n < p \leq 2n$}
\author{Haaroon Hussain \& Wanja Stämpfli}
\date{\today}

\begin{document}

% -------------------------------------------------------------------
% Title page
% -------------------------------------------------------------------
\begin{frame}
  \titlepage
\end{frame}

% -------------------------------------------------------------------
% Definition
% -------------------------------------------------------------------

\begin{frame}[t]{Definition}
 \vspace{1em}
  \bertrandpostulat
  
  \vfill
  \begin{center}
   \begin{tikzpicture}[x=1\textwidth/86, y=1cm]
     \draw[->] (0,0) -- (86,0);
     \foreach \x in {3, 5, 7, 13, 23, 43, 83} {
       \draw[PrimaryBlue, thick] (\x, -0.15) -- (\x, 0.15) node[above, font = \scriptsize] {\x};
       \draw[AccentOrange, thick] (30, -0.2) -- (30, 0.2) node[above, font = \scriptsize] {$n$};
       \draw[AccentOrange, thick] (60, -0.2) -- (60, 0.2) node[above, font = \scriptsize] {$2n$};      
       }
     \node[below right] at (86,0) {};
   \end{tikzpicture}
  \end{center}
  \vfill
\end{frame}

% -------------------------------------------------------------------
% History
% -------------------------------------------------------------------

% highlight Erdös to indicate that we look at his proof

\begin{frame}{History}
  \begin{itemize}\setlength{\itemsep}{1.5em}
        \item \textbf{Bertrand (1845)}: Conjectured the statement after verifying it for $n \leq 3,000,000$.
        \item \textbf{Chebyshev (1852)}: Provided the first rigorous proof using factorial and prime properties.
        \item \textbf{Ramanujan (1919)}: Shorter proof using Sterling Formula.
        \item \textbf{Erdős (1932)}: Elegant combinatorial proof using no calculus.

    \end{itemize}

\end{frame} 

% -------------------------------------------------------------------
% General Approach - Overview 1
% -------------------------------------------------------------------

\begin{frame}[t]
  \frametitle{%
    General Approach % Left part
    \hfill % Push the box to the right
    \raisebox{2pt}[\dimexpr\height-1pt][0pt]{% Fine-tune vertical alignment if needed
      \header{For every $n \geq 1$ there is some prime $p$ with $n < p \leq 2n$}
    }%
  }
  \vspace{1em}
 
  % --- Content for Slide 1 ---
  \bertrandpostulat
   \vfill
   \begin{center}
     \begin{itemize}
       \setlength{\itemsep}{2em} % Apply spacing here
       \item Choose $f(n)$ that contains these primes $p$
       \item Find a lower bound $f_{\min}(n)\le f(n)$
       \item If no primes in $(n,2n]$ then $f_{\min}(n) \le f(n)$ only holds for $n < N$
       \item Verify for $n < N$
     \end{itemize}    
   \end{center}
   \vfill
 \end{frame}
% -------------------------------------------------------------------
% General Approach -  choosing f(n)
% -------------------------------------------------------------------

\begin{frame}[t]
 \frametitle{%
   General Approach % Left part
   \hfill % Push the box to the right
   \raisebox{2pt}[\dimexpr\height-1pt][0pt]{% Fine-tune vertical alignment if needed
     \header{For every $n \geq 1$ there is some prime $p$ with $n < p \leq 2n$}
   }%
 }
 \vspace{1em}


 % --- Content for Slide 2 ---
 \only<1-2>{
  \begin{itemize}
    \item Choose $f(n)$ that contains these primes $p$
  \end{itemize}
     \begin{block}{}
       $$
       \binom{2n}{n}=\frac{(2n)!}{n!n!}
       $$
       \only<2>{\textbf{Example: $n=4$}
       \vspace{1em}
       $$
       \binom{8}{4}= \frac{\textcolor{red}{5} \cdot 6 \cdot \textcolor{red}{7} \cdot 8}{2 \cdot 3 \cdot 4 } = 
       \frac{\textcolor{red}{5} \cdot (2 \cdot 3) \cdot \textcolor{red}{7} \cdot 2^3}{2 \cdot 3 \cdot 2^2 } = \textcolor{red}{5} \cdot \textcolor{red}{7} \cdot 2
       $$}
     \end{block}
 } % End of content for slide 2

\end{frame}
% -------------------------------------------------------------------
% General Approach - Lowerbound
% -------------------------------------------------------------------

\begin{frame}[t]
  \frametitle{%
    General Approach % Left part
    \hfill % Push the box to the right
    \raisebox{2pt}[\dimexpr\height-1pt][0pt]{% Fine-tune vertical alignment if needed
      \header{For every $n \geq 1$ there is some prime $p$ with $n < p \leq 2n$}
    }%
  }
  \vspace{1em} 
   \begin{itemize}
     \item Find a lower bound $f_{\min}(n)\le \binom{2n}{n}$
   \end{itemize}
      \vfill
      \begin{center}
        \only<1>{
          \PascalTriangle{6}
        }
        \only<2>{
          \PascalTriangleHighlight{6}
        }
        \only<3>{
          \PascalTriangleHighlightMid{6}
        }
        \only<4>{
         \PascalHighlightSum{6}
        }
        \only<5>{
          $$
          2^{2n} \overset{(1)}{\le} 2n \cdot \binom{2n}{n} \implies f_{min}(n) = \frac{2^{2n}}{2n}
          $$
          
        }
      \end{center}
      \vfill
      \only<5>{\scriptsize (1): $\binom{2n}{0} + \binom{2n}{2n} \le \binom{2n}{n}$}
 \end{frame}
 % -------------------------------------------------------------------
% General Approach - Overview 2
% -------------------------------------------------------------------

\begin{frame}[t]
  \frametitle{%
    General Approach % Left part
    \hfill % Push the box to the right
    \raisebox{2pt}[\dimexpr\height-1pt][0pt]{% Fine-tune vertical alignment if needed
      \header{For every $n \geq 1$ there is some prime $p$ with $n < p \leq 2n$}
    }%
  }
  \vspace{1em}
 
  % --- Content for Slide 1 ---
   \vfill
   \begin{center}
     \begin{itemize}
       \setlength{\itemsep}{2em} % Apply spacing here
       \item $\binom{2n}{n}$ contains all primes in $(n, 2n]$
       \item $\frac{2^{2n}}{2n} \le \binom{2n}{n}$
       \item If no primes in $(n,2n]$ then $\frac{2^{2n}}{2n} \le \binom{2n}{n}$ only holds for $n < N$
       \item Verify for $n < N$
     \end{itemize}
   \end{center}
   \vfill
 \end{frame}
% -------------------------------------------------------------------
% Contradiction slide
% -------------------------------------------------------------------
\begin{frame}{Contradiction Slide}
  \textbf{Contradiction:}

  Assume $\exists n$ such that there is no prime $p$ with $n<p\le 2n$, 
  then show that $\binom{2n}{n}$ is smaller than some fixed lowerbound.
  
  
  $$
  Lowerbound \le \binom{2n}{n}= \prod_{p_i \le n}p_i^{r_i} \cdot \prod_{n < p_i \le 2n}p_i^{r_i}
  $$
  Then our assumption should lead to 
  \corecontent {
  $$ Lowerbound \; \textcolor{red}{>} \prod_{p_i \le n}p_i^{r_i} \cdot \underbrace{\prod_{n < p_i \le 2n}p_i^{r_i}}_{=0}
  $$}

  \end{frame}


% -------------------------------------------------------------------
%Proof Idea - Summary
% -------------------------------------------------------------------
\begin{frame}{Proof Idea - Summary}

  \corecontent {
    $$ \frac{4^n}{2n} \; \le \binom{2n}{n} = \prod_{p_i \le n}p_i^{r_i} \cdot \prod_{n < p_i \le 2n}p_i^{r_i}
    $$}
    \centralquestion{What can we say about how often the primes appear in $\binom{2n}{n}$ i.e how large are the $r_i$'s ?}
  \end{frame}
% -------------------------------------------------------------------
%Legendre - Theorem
% -------------------------------------------------------------------
\begin{frame}\frametitle{Legendre's Theorem}
  \uncover<1->{
\begin{bertrandbox}
  \(n!\) contains the prime factor \(p\) exactly $\sum_{k \ge 1} \left\lfloor \frac{n}{p^k} \right\rfloor$ times 
\end{bertrandbox}
  }
\uncover<1->{
\textbf{Intuition:}
}
\begin{itemize}
    \item<2-> Exactly \(\left\lfloor \frac{n}{p} \right\rfloor\) of the factors from 
    \(n! = 1 \cdot 2 \cdot 3 \cdots n\) are divisible by \(p\) since $p, 2p, 3p, …, \left\lfloor \frac{n}{p} \right\rfloor \cdot p \leq n$.
    
    \item<3-> iterate over $k$ because higher powers $p^2$, $p^3$ etc. contribute additional factors of $p$ that must be counted separately.
  \end{itemize}
\end{frame}

% -------------------------------------------------------------------
% Legendre - Example
% -------------------------------------------------------------------

\begin{frame}\frametitle{Legendre's Theorem - Example}
  \begin{bertrandbox}
    \(n!\) contains the prime factor \(p\) exactly $\sum_{k \ge 1} \left\lfloor \frac{n}{p^k} \right\rfloor$ times 
  \end{bertrandbox}
  \textbf{Example:} $n=8$ and $p=2$
  \begin{itemize}
      \item[$k=1$:]<1-> \[
1 \cdot \textcolor{red}{2} \cdot 3 \cdot \textcolor{red}{2^2} \cdot 5 \cdot (\textcolor{red}{2} \cdot 3) \cdot 7 \cdot \textcolor{red}{2^3} \implies \left\lfloor \frac{n}{p} \right\rfloor = 4
\]
      \item[$k=2$:]<2-> \[
1 \cdot 2 \cdot 3 \cdot \textcolor{orange}{2^2} \cdot 5 \cdot (2 \cdot 3) \cdot 7 \cdot \textcolor{orange}{2^3} \implies \left\lfloor \frac{n}{p^2} \right\rfloor = 2
\]
      \item[$k=3$:]<3-> \[
1 \cdot 2 \cdot 3 \cdot 2^2 \cdot 5 \cdot (2 \cdot 3) \cdot 7 \cdot \textcolor{violet}{2^3} \implies \left\lfloor \frac{n}{p^3} \right\rfloor = 1
\]
    \end{itemize}
  \end{frame}

% -------------------------------------------------------------------
%Legendre's Theorem - Application
% -------------------------------------------------------------------
%keep the Legendre thm for better overview
%color the terms to see the correspondence better
\begin{frame}\frametitle{Legendre's Theorem - Application}
  \uncover<1->{
  \centralquestion{How often does prime \(p\) appear in \(\frac{(2n)!}{n!\,n!}\)?}
  }
  \uncover<2->{
  \begin{center}
  \text{\small Accounting for cancellation from the fraction yields:}
  \[
    \sum_{k \ge 1}
      \Bigl(\left\lfloor \tfrac{2n}{p^k}\right\rfloor 
      \;-\;
      2\left\lfloor \tfrac{n}{p^k}\right\rfloor\Bigr)
  \]
  \end{center}
  }
  
  \uncover<3->{
  \begin{center}
  \text{\small Because \(\lfloor x \rfloor < x\) and \(\lfloor x \rfloor > x - 1\), each summand satisfies:}
  \[
    \left\lfloor \tfrac{2n}{p^k}\right\rfloor
    \;-\;
    2\left\lfloor \tfrac{n}{p^k}\right\rfloor
    \;\textcolor{red}{<}\;
    \tfrac{2n}{p^k}
    \;-\;
    2\Bigl(\tfrac{n}{p^k} - 1\Bigr)
    \;=\; 2
  \]
  \end{center}
  }
  
  \uncover<4->{
    \begin{center}
      \text{\small Each term is 0 whenever $k > r$ and at most $1$ otherwise, thus:}
    \end{center}
  \begin{resultbox}
  \[
    \sum_{k \ge 1}
      \Bigl(\left\lfloor \tfrac{2n}{p^k}\right\rfloor
      \;-\;
      2\left\lfloor \tfrac{n}{p^k}\right\rfloor\Bigr)
    \;{\le}\;
    \max\{\,r \;\mid\; p^r \le 2n\,\}.
  \]
  \end{resultbox}
  }
  \end{frame}


% -------------------------------------------------------------------
%Legendre's Theorem - Observations
% -------------------------------------------------------------------
\begin{frame}\frametitle{Legendre's Theorem - Observations}

  % 1) Shown from overlay #1 onward
  \uncover<1->{
    \begin{resultbox}
      prime \(p\) appears in \(\frac{(2n)!}{n!\,n!}\) at most 
      \(\max\{\,r \mid p^r \le 2n\}.\)
    \end{resultbox}
  }
  
  \begin{itemize}
    % 2) Bullet #1: revealed on overlay #2, remains afterward
    \uncover<1->{
      \item Largest power of \(p\) in the factorization cannot exceed \(2n\).
    }
  
    % 3) Bullet #2: revealed on overlay #3, remains afterward
    \uncover<2->{
      \item Primes satisfying $p > \sqrt{2n}$ appear at most once.
    }
  
    % 4) Bullet #3: revealed on overlay #4, remains afterward
    \uncover<3->{
      \item primes $p$ that satisfy $\frac{2}{3}n < p \le n$ don't appear at all.
    }
  \end{itemize}
  
  % -- Factorization at the bottom, changing each time --
  
  % Shown on overlay #1 ONLY
  \only<1>{
    \corecontent{
      \[
        \frac{(2n)!}{n!\,n!}
        \;\le\;
        \prod_{\substack{p \le 2n}} 2n
      \]
    }
  }
  
  % Shown on overlay #2 ONLY
  \only<2>{
    \corecontent{      
      \[
      \frac{(2n)!}{n!n!}
      \;\le\;
      \prod_{p \le \sqrt{2n}} 2n
      \;\cdot\;
      \prod_{\sqrt{2n} < p \le 2n} p
    \]}
  }
  
    % Shown on overlay #3 ONLY
    \only<3>{
      \corecontent{      
        \[
        \frac{(2n)!}{n!n!}
        \;\le\;
        \prod_{p \le \sqrt{2n}} 2n
        \;\cdot\;
        \prod_{\sqrt{2n} < p \le \frac{2}{3}n} p
        \;\cdot\;
        \prod_{n < p \le 2n} p
      \]}
    }
  
  \end{frame}
  
% -------------------------------------------------------------------
% -------------------------------------------------------------------

\begin{frame}\frametitle{Extensiosnsssss}
  Weil:

  \begin{itemize}
    \item Für \(3p > 2n\) (und \(n \ge 3\) und damit \(p \ge 3\)) sind \(p\) 
          und \(2p\) die einzigen Vielfachen von \(p\), die im Zähler von 
          \(\frac{(2n)!}{n!\,n!}\) vorkommen, während wir zwei 
          \(p\)-Faktoren im Nenner haben.
  \end{itemize}
\end{frame}

\begin{frame}{Abstract Approach}
 \bertrandpostulat
   \begin{itemize}[<+->]          % automatischer Folienaufbau
     \item Choose $f(n)$ that contains these primes $p$
     \item Find a lower bound $f_{\min}(n)\le f(n)$
     \item Assume no prime $p$ in $(n,2n]$ and show
           $f_{\min}(n) > f(n)$ for $n\ge N$
     \item Verify for $n\le N$
   \end{itemize}
\end{frame}



\begin{frame}\frametitle{Product of prime numbers}

\begin{bertrandbox} \[\prod_{p \leq x} p \leq 4^x - 1 \quad \text{for all real } x \geq 2\] 
\end{bertrandbox}

\begin{itemize}
    \uncover<2->{\item Let \( q \) be the largest prime \( \leq x \), then:
    \[\prod_{p \leq x} p = \prod_{p \leq q} p \quad \text{and} \quad 4^q - 1 \leq 4^x - 1\]}
    
    \uncover<3->{\item It suffices to prove for \( x = q \) (a prime).}
    \uncover<4->{\item We proof now for all primes by induction.}
\end{itemize}

\end{frame}

% -------------------------------------------------------------------
\begin{frame}\frametitle{Inductive Base Case}

\begin{itemize}
    \uncover<1->{\item Base case: For \( q = 2 \), we have \( 2 \leq 4^{2-1}= 4 \).}
    \uncover<2->{\item We are left with only odd primes \( q = 2m + 1 \).}
    \uncover<3->{\item Inductive step for arbitrary odd prime:}
\end{itemize}

\end{frame}

% -------------------------------------------------------------------
\begin{frame}\frametitle{Inductive Step}

$$
  \prod_{p \leq 2m+1} p = \prod_{p \leq m+1} p \prod_{m+1 < p \leq 2m+1} p
  \uncover<3->{\leq 4^m  \prod_{m+1<p\leq 2m+1} p }
%   \frac{2m + 1}{m} actual bound of above
%  \leq 4^{m^2} 2^m 
%  = 4^{2m}
  
$$

\uncover<2->{Using the induction hypothesis: \( \prod_{p \leq m+1} p \leq 4^m \).}

\end{frame}

% -------------------------------------------------------------------
\begin{frame}\frametitle{Bounding the Prime Product}

\centralquestion{Can we bound this $\prod_{m+1 < p \leq 2m+1} p $ ?}


\begin{itemize}
    \item Yes, \( \frac{(2m+1)!}{m! (m+1)!} \) is an integer and contains all prime numbers between $m+1$ and $2m+1$.
    \uncover<2->{\item Follows since both denominator have factors less than  $m+1$.}
\end{itemize}
\uncover<2->{\begin{bertrandbox}
     $$\prod_{p \leq 2m+1} p = \prod_{p \leq m+1} p \prod_{m+1 < p \leq 2m+1} p\leq 4^m  \binom{2m+1}{m}$$
\end{bertrandbox}}
\end{frame}

% -------------------------------------------------------------------
\begin{frame}\frametitle{Binomial Bound}

\centralquestion{ $\binom{2m+1}{m}\leq 2^{2m}$ ? }
\uncover<2->{
$$
   \binom{2m+1}{m}\cdot 2\leq  \sum_{k=0}^{2m+1} \binom{2m+1}{k} 
  = 2^{2m+1}
$$ }

\uncover<3->{
Since $$\binom{2m+1}{m} = \binom{2m+1}{m+1} $$}

\end{frame}

\begin{frame}\frametitle{Conclusion}
Plugging everything together we get:
\uncover<2->{\begin{bertrandbox}
     $$\prod_{p \leq 2m+1} p = \prod_{p \leq m+1} p \prod_{m+1 < p \leq 2m+1} p\leq 4^m  \cdot 4^m = 4^{2m}$$
\end{bertrandbox}}
\end{frame} 
% -------------------------------------------------------------------

\begin{frame}\frametitle{Temporary results}
\corecontent{      
        \[
        4^n
        \;\le\;
        \prod_{p \le \sqrt{2n}} 2n
        \;\cdot\;
        \prod_{\sqrt{2n} < p \le \frac{2}{3}n} p
        \;\cdot\;
        \prod_{n < p \le 2n} p
      \]}
\uncover<2->{      \corecontent{      
        \[
       (2n)^{1+\sqrt{2n}}
        \;\cdot\;
        4^{ \frac{2}{3}n}
        \;\cdot\;
        \prod_{n < p \le 2n} p
      \]}}
\end{frame}

\begin{frame}\frametitle{Proof by Contradiction}
Assume no prime $p$ between $n < p \leq 2n$ 
\corecontent{      
        \[
        4^n
        \;\le\;
        (2n)^{1+\sqrt{2n}}
        \;\cdot\;
        4^{ \frac{2}{3}n}
      \]}
\uncover<2->{  \corecontent{      
        \[
        4^{\frac{1}{3}n}
        \;\le\;
        (2n)^{1+\sqrt{2n}}
      \]}}
\end{frame}

\begin{frame}\frametitle{Intuition}
\begin{figure}
    \centering
    \includegraphics[width=1\linewidth]{contra.png}
    \caption{Plot of the \[
        4^{\frac{1}{3}n}
        \;\le\;
        (2n)^{1+\sqrt{2n}}
      \]}
    \label{fig:enter-label}
\end{figure}
\end{frame}


\begin{frame}\frametitle{Rigorous proof}

Using \( a + 1 < 2^a \) for \( a \geq 2 \), we get:
\[
2n = \left( \sqrt[6]{2n} \right)^6 < \left( \lfloor \sqrt[6]{2n} \rfloor + 1 \right)^6 < 2^{6 \lfloor \sqrt[6]{2n}} \rfloor \leq 2^{6 \sqrt[6]{2n}}.
\]
\end{frame}

\begin{frame}{Verification for Small Values of $n$}

\vspace{0.3cm}

\begin{center}
% First number line
\begin{tikzpicture}[x=1\textwidth/70, y=1cm]
    \draw[->] (0,0) -- (60,0);

    % First number line primes
    \foreach \i/\x in {0/2, 1/3, 2/5, 3/7, 4/13, 5/23, 6/43, 7/83} {
        \ifodd\i
            \draw[blue, thick] (\x, -0.15) -- (\x, 0.15) node[below=6pt, font=\scriptsize, blue] {\x};
        \else
            \draw[blue, thick] (\x, -0.15) -- (\x, 0.15) node[above=6pt, font=\scriptsize, blue] {\x};
        \fi
    }
\end{tikzpicture}

\vspace{1cm}

% Second number line
\begin{tikzpicture}[x=1\textwidth/70, y=1cm]
    \draw[->] (0,0) -- (60,0);

    % Second number line primes
    \foreach \i/\x in {0/163, 1/317, 2/631, 3/1259, 4/2503, 5/4001} {
        \ifodd\i
            \draw[blue, thick] (\x/80, -0.15) -- (\x/80, 0.15) node[below=6pt, font=\scriptsize, blue] {\x};
        \else
            \draw[blue, thick] (\x/80, -0.15) -- (\x/80, 0.15) node[above=6pt, font=\scriptsize, blue] {\x};
        \fi
    }
\end{tikzpicture}
\end{center}

\vspace{0.5cm}

\begin{bertrandbox}
\textbf{Conclusion:} Every interval $\{y : n < y \leq 2n\}$, with $n \leq 4000$, contains one of these 14 primes.
\end{bertrandbox}

\end{frame}

\begin{frame}\frametitle{Bounding \( 2^{2n} \) and Conclusion}

For \( n \geq 50 \) (hence \( 18 < 2\sqrt{2n} \)), we obtain:
\[
2^{2n} \leq (2n)^{3(1+\sqrt{2n})} < 2^{6\sqrt{2n}(18+18\sqrt{2n})} < 2^{20 \sqrt{2n} \sqrt{2n}} = 2^{20(2n)^{2/3}}.
\]
This implies \( (2n)^{1/3} < 20 \), and thus \( n < 4000 \).
\end{frame}


\begin{frame}\frametitle{Further reads
}

Is there a prime number between every consecutive perfect squares?
\end{frame}
\end{document}